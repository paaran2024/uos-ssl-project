\section{Experimental Results}
\subsection{Quantitative Results}

%%PSNR/SSIM
%%Teacher vs Student 비교 테이블


%% catanet 논문 table7 첨부함. xs결과나오면 추가해서 비교하면 좋을듯?
%% 필요없음 지우셈요

\begin{table}[t]
    \centering
    \caption{Performance comparison of CATANet variants and lightweight SR models on Set5 and Set14. 
    Red values denote the results of our CATANet-XS model.}
    \label{tab:catanet_baseline_reduced}
    \setlength{\tabcolsep}{5pt}
    \renewcommand{\arraystretch}{1.2}
    \begin{tabular}{l c c c}
        \hline
        \textbf{Method} & \textbf{Params} & \textbf{Set5} & \textbf{Set14} \\
        \hline
        ATD-L & 484K & 38.23 & 33.94 \\
        SMFANet+ & 480K & 38.18 & 33.82 \\
        \textbf{CATANet-L} & 477K & 38.28 & 33.99 \\
        \textbf{CATANet-XS (ours)} & \textcolor{red}{---} & \textcolor{red}{---} & \textcolor{red}{---} \\
        \hline
        ATD-M & 300K & 38.14 & 33.78 \\
        DITN-Tiny & 367K & 38.00 & 33.70 \\
        \textbf{CATANet-M} & 306K & 38.17 & 33.94 \\
        \textbf{CATANet-XS (ours)} & \textcolor{red}{---} & \textcolor{red}{---} & \textcolor{red}{---} \\
        \hline
        ATD-S & 229K & 38.07 & 33.74 \\
        SAFMN & 228K & 38.00 & 33.54 \\
        SeemoRe-T & 220K & 38.06 & 33.62 \\
        \textbf{CATANet-S} & 230K & 38.13 & 33.80 \\
        \textbf{CATANet-XS (ours)} & \textcolor{red}{---} & \textcolor{red}{---} & \textcolor{red}{---} \\
        \hline
    \end{tabular}
\end{table}



\subsection{Qualitative Results}

%%% 복원 이미지 전/후 사진 첨부
%% 차이 분석..?