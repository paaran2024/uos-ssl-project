\section{논의}

\subsection{자동 프루닝이 수동 설계를 이길 수 있는가?}
우리의 처음 동기는 단순했다: 자동으로 생성된 아키텍처가 수동으로 설계된 것보다 나을 수 있을까? 결과는 그렇다고 말하지만, 조건이 붙는다.

CATANet-XS는 Set5에서 38.15 dB를 달성하여 CATANet-S의 38.13 dB를 약간 앞섰다. 이것이 의미 있는 이유는 CATANet-S가 원 저자들에 의해 신중하게 수동 튜닝된 반면, 우리 모델은 프루닝을 통해 CATANet-L에서 자동으로 유도되었기 때문이다. 수동 하이퍼파라미터 튜닝(예: dim=36, block\_num=4 설정)이 최적의 파라미터 할당을 보장하지 않는다는 우리의 가설이 검증되었다고 생각한다.

그러나 성능 차이가 미미하다는 점(0.02 dB)을 인정해야 한다. 이 차이가 실질적으로 유의미한지는 의문이다. 확실히 말할 수 있는 것은 자동 프루닝이 최소한 수동 설계 품질에 필적하면서 더 큰 교사 모델의 지식을 상속받는다는 점이다.

\subsection{왜 Output KD가 가장 잘 동작하는가}
처음에는 FaKD가 정교한 공간 친화도 전이 메커니즘을 고려할 때 더 단순한 접근법보다 나을 것으로 예상했다. 놀랍게도 Output KD가 가장 좋은 결과(38.15 dB)를 달성했고, FaKD(38.14 dB)와 Feature KD(38.08 dB)가 뒤따랐다.

초해상도가 근본적으로 픽셀 수준 재구성 작업이기 때문에 이런 결과가 나왔다고 생각한다. 최종 출력이 직접적으로 PSNR을 결정하므로, 학생이 교사의 출력을 픽셀 단위로 일치시키도록 강제하는 것이 가장 직접적인 감독을 제공한다. FaKD의 친화도 행렬은 구조적 관계를 포착하지만, 출력 자체가 이미 일치되고 있을 때는 중복될 수 있다.

Feature KD의 낮은 성능은 예상 밖이었다. CATANet-L과 프루닝된 모델의 중간 특징 공간이 서로 다른 분포를 가지고 있어서, 추가적인 정렬 레이어 없이는 직접적인 특징 매칭이 최적이 아닐 수 있다고 추측한다.

\subsection{모바일 배포: 배운 교훈}
CATANet-XS를 모바일 기기에 배포하면서 벤치마크 수치에서는 드러나지 않는 실제적인 문제들이 드러났다.

320$\times$320 입력 모델을 사용한 초기 접근법은 iOS에서 메모리 오버플로우 오류(3GB 제한 초과)로 충돌을 일으켰다. 이로 인해 64$\times$64 타일을 사용한 타일 기반 처리를 구현해야 했다. 메모리 문제는 해결되었지만 타일 경계에서 눈에 보이는 이음새 아티팩트가 발생했고, 8픽셀 오버랩 블렌딩으로 완화했다.

또한 부드러운 영역에서 산발적인 초록색 얼룩 아티팩트를 관찰했다(Fig.~\ref{fig:artifact_analysis}). 이것은 데스크톱 추론에서는 나타나지 않아서, TorchScript 변환이나 모바일 부동소수점 정밀도 차이에서 비롯된 것으로 보인다. 이것은 여전히 미해결 문제로 남아있다.

\begin{figure}[h]
    \centering
    \begin{tabular}{cc}
        \includegraphics[width=0.45\linewidth]{fig7_input.png} &
        \includegraphics[width=0.45\linewidth]{fig7_output.jpeg} \\
        (a) 입력 & (b) 출력 ($\times$2) \\
    \end{tabular}
    \caption{모바일 애플리케이션에서의 시각적 비교.}
    \label{fig:visual_comparison}
\end{figure}

\begin{figure}[h]
    \centering
    \begin{tabular}{cc}
        \includegraphics[width=0.45\linewidth]{fig8_zoom_input.jpeg} &
        \includegraphics[width=0.45\linewidth]{fig8_zoom_output.jpeg} \\
        (a) 입력 (확대) & (b) 출력 (확대) \\
    \end{tabular}
    \caption{아티팩트 분석: 모바일 추론 후 부드러운 영역에서 나타나는 초록색 얼룩.}
    \label{fig:artifact_analysis}
\end{figure}

\subsection{한계점}
몇 가지 한계점을 인정해야 한다. 첫째, 평가가 Set5에만 집중되었다. 더 광범위한 벤치마크(Set14, Urban100, Manga109)가 더 포괄적인 검증을 제공할 것이다. 둘째, 실시간 비디오 처리(24 fps)라는 원래 목표를 달성하지 못했다. 타일 기반 접근법은 상당한 오버헤드를 추가하며, 현재 모바일 하드웨어로는 실용적인 해상도에서 실시간 SR을 유지할 수 없다. 셋째, 프루닝된 모델은 실제 파라미터 감소를 위해 여전히 명시적인 재구축이 필요하다. 현재 구현은 물리적으로 작은 가중치가 아닌 희소 마스크를 사용한다.

\subsection{향후 방향}
두 가지 유망한 방향이 보인다. 첫째, 하이브리드 손실 함수를 통해 Output KD의 안정성과 FaKD의 구조적 감독을 결합하면 더 좋은 결과를 얻을 수 있을 것이다. 둘째, INT8 양자화를 조사하면 메모리 사용량을 더 줄이고 잠재적으로 더 큰 타일 크기를 가능하게 하여 경계 아티팩트를 줄이면서 처리량을 개선할 수 있을 것이다.
