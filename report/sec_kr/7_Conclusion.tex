\section{결론}

우리는 단순한 질문에 답하고자 했다: 자동으로 압축된 모델이 수동으로 설계된 것과 대등하거나 더 나을 수 있는가? CATANet-XS 실험 결과는 그렇다고 말한다.

CATANet-L에 OPTIN 기반 프루닝을 적용하고 지식 증류를 통해 성능을 복구하여, FLOPs가 54.96\% 감소된 CATANet-XS를 만들었다. 결과 모델은 Set5에서 38.15 dB를 달성하여 수작업으로 설계된 CATANet-S(38.13 dB)를 약간 앞섰다. 수동 아키텍처 설계가 최적의 파라미터 활용을 보장하지 않는다는 처음 가설이 검증되었다.

테스트한 세 가지 KD 전략 중 Output KD가 가장 단순함에도 불구하고 가장 효과적이었다. 초해상도 작업의 픽셀 수준 특성상 직접적인 출력 매칭이 중간 특징 정렬보다 강한 감독을 제공하기 때문이라고 생각한다.

Flutter 기반 모바일 애플리케이션을 통해 실제 배포도 시연했다. 이 과정에서 벤치마크 성능이 배포 성공으로 직접 이어지지 않는다는 것을 알게 되었다. 메모리 제약, 정밀도 차이, 처리 오버헤드가 타일 기반 추론과 오버랩 블렌딩 같은 엔지니어링 솔루션을 요구하는 문제를 야기했다.

우리 연구에는 명확한 한계가 있다. Set5에서만 평가했고, 실시간 비디오 처리를 달성하지 못했으며, 모바일 추론에서 설명되지 않는 아티팩트가 관찰되었다. 이것들은 향후 조사를 위한 미해결 문제로 남아있다.

이러한 한계에도 불구하고, 이 연구가 경량 트랜스포머 기반 SR 모델을 압축하는 실행 가능한 경로를 보여준다고 믿는다. 프루닝-증류 파이프라인은 CATANet에 특화된 것이 아니며 다른 아키텍처에도 적용될 수 있다. 온디바이스 초해상도를 연구하는 실무자들에게 KD 전략 선택과 모바일 배포 문제에 대한 우리의 발견이 유용하기를 바란다.
